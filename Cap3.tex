\chapter{Sistemas de detecci\'on de uso radiol\'ogico}
% \chapterauthors{M. Valente \& P. Pérez
% \chapteraffil{Carnegie Mellon University, Pittsburgh, Pennsylvania}
% }

El \textit{Cap\'itulo} \ref{CapIII} presenta descripciones breves respecto de los principios de funcionamiento y detalles t\'ecnicos 
de los sistemas de detecci\'on de radiaci\'on m\'as com\'unmente empleados en el \'ambito de radiodiagn\'ostico.

En l\'ineas generales, los detectores de radiaci\'on presentan similitudes en cuanto a su comportamiento.
%

%
Los efectos de interacci\'on entre la radiaci\'on y la materia son la base que determina de modo un\'ivoco las propiedades de los sistemas de
detecci\'on. En particular, el tipo de material del detector depende propiamente de la clase de radiaci\'on as\'i como de la informaci\'on que 
es necesario recavar.
%

%
La operatividad de los sistemas de detecci\'on deben contar con las siguientes etapas:

\begin{itemize}
 \item Ingreso de la radiaci\'on al sistema de detecci\'on.
 \item Interacci\'on de la radiaci\'on con el ``material sensible'' que constituye el sistema de detecci\'on.
 \item Efectos por interacci\'on de la radiaci\'on con el material sensible: p\'erdida de toda o parte de su energ\'ia cin\'etica por 
 medio de transferencias a los electrones de los \'atomos del material sensible.
 \item Producci\'on de corrientes de electrones (de energ\'ias relativamente bajas).
 \item Recolecci\'on de la corriente de electrones.
 \item An\'aslisis mediante circuito electr\'onico.
 \item Procesamiento con dispositivos digitales (opcional).
\end{itemize}

En t\'erminos del tipo de radiaci\'on a detectar, puede mencionarse, esquem\'aticamente:


\begin{description}
 \item[Determinaci\'on del tipo de part\'icula] Identificar el tipo de part\'icula (que resulta cr\'itico en el caso de un campo mixto,
 como ocurre en procesos nucleares) es necesario utilizar materiales sensibles en los que ya sea la carga o la masa de cada tipo de
 part\'icula pueda generar efectos distintivos.
 \item[Tiempo de emisi\'on de radiaci\'on] Medir el tiempo en el que la radiaci\'on fue emitida rquiere de materiales sensibles en los 
 que sea posible una r\'apida recolecci\'on de los pulsos de corriente de electrones producidas por las interacciones.
 \item[Energ\'ia de la radiaci\'on] Determinar la energ\'ia de la radiaci\'on implica utilizar detectores en los que la amplitud de los 
 pulso detectados resulte proporcional a la energ\'ia de la radiaci\'on que provoc\'o el pulso. El material sensible debe ser 
 de alto n\'umero de electrones disponibles de modo que se minimicen p\'erdidas y fluctuaciones.
 \item[Polarizaci\'on de la radiaci\'on] La medici\'on del spin o la polarizaci\'on requiere de detector capaces de separar los diferentes 
 estados de polarizaci\'on de la radiaci\'on. En general, no alcanza solo con materiales sensibles, sino que debe acudirse al dise\~no de 
 arreglos espec\'ificos para detecci\'on.
 \item[Tasa de conteo de flujo] Para determinaciones de alta tasa de conteo, es necesario emplear detectores de r\'apida 
 recuperaci\'on capaces de reiniciar el conteo de eventos sucesivos. Contrariamente, para mediciones de tasas de conteo muy bajas,
 lo m\'as importante es la minimizaci\'on del ruido de fondo.
\end{description}

\section{Procesos para la detecci\'on de radiaci\'on electromagn\'etica}
% \markboth{Intr. proc. im\'agenes radiol\'ogicas \'ambito m\'edico \ \textbf{M\'ODULO III}}{ESPECIALIDAD III \ \textbf{M\'ODULO III}}
\label{CapIII_1}

Los fotones (restringiendo al campo de aplicaci\'on en radiodiagn\'ostico, refiere a rayos X y $\gamma$) interactuan con la materia 
por medio de diferentes tipos de procesos: \textit{scattering} Compton, creaci\'on de pares y absorci\'on fotoel\'ectrica.

A continuaci\'on se incluye Una descripci\'on brev\'isima estos procesos:

\begin{center}
\underline{\textit{Scattering} Compton}
\end{center}

El \textit{scattering} Compton es el proceso por el cual un fot\'on incidente cambia el estado de fase, modificando potencialmente 
direcci\'on de movimiento $\vec{\Omega}$ y energ\'ia cin\'etica $E$ por interacciones con electrones de los orbitales at\'omicos, los que 
inicialmente pueden considerarse pr\'acticamente libres\footnote{Las energ\'ias de ligadura t\'ipicas son mucho menores a 
las del fot\'on incidente} adquieren casi toda la energ\'ia cin\'etica liberada por el fot\'on incidente. En este sentido, aproximando por 
electr\'on en reposo y libre se aplica la conservaci\'on de momento y energ\'ia para describir los cambios de fase.
%

%
\begin{center}
\underline{Producci\'on de pares}
\end{center}

Este proceso refiere a la interacci\'on de un fot\'on incidente energ\'etico con la materia de modo de producir pares electr\'on-positr\'on
como consecuencia de acoplamiento con el campo at\'omico. La energ\'ia cin\'etica es cedida para el equivalente en masa de par 
part\'icula-antipart\'icula, y eventual sobrante es transferido como energ\'ia coin\'etica a las part\'iculas creadas.
%

%
Por lo tanto, existe un valor umbral para la energ\'ia por encima del cual es posible el efecto: 
$E_{umbral} = 2 \, m_{e} c^{2} = 1.022 \, MeV$.

\begin{center}
\underline{Absorci\'on fotoel\'ectrica}
\end{center}

El fot\'on incidente es absorbido por parte del \'atomo de modo que uno de los electrones at\'omicos, denominado fotoelectr\'on es 
liberado a expensas de la energ\'ia cin\'etica adquirida. Los electrones libres no pueden absorber fotones para cumplir simult\'aneamente 
con la conservaci\'on de la energ\'ia y el momento, motivo por el cual no se produce este efecto para electrones libres. La energ\'ia 
cin\'etica del electr\'on de ionizaci\'on (liberado) equivale a la energ\'ia del fot\'on incidente menos la energ\'ia de ligadura del 
electr\'on eyectado.
%

%
La determinaci\'on de la probabilidad de absorci\'on de un fot\'on por efecto fotoel\'ectrico muestra algunas caracter\'isticas 
espec\'ificas, como que es mayor para para energ\'ias bajas\footnote{Energ\'ias menores a 100keV, aproximadamente.}, aumenta 
significativamente seg\'un el n\'umero at\'omico $Z$ y disminuye seg\'un aumente la energ\'ia del fot\'on incidente $E$.
%

%
A partir de los procesos mencionados, se propone una cantidad para intengrar los efectos netos denominada coeficiente de 
atenuaci\'on m\'asico, el cual se describe del siguiente modo: Se considera un haz perfectamente colimado de fotones de energ\'ia $E$
producidos por una fuente $S$ e incidiendo sobre un material de n\'umero at\'omico $Z$ y espesor $d$ (a lo largo del \textit{path}).
%
Por lo tanto, en los procesos de interacci\'on, los fotones del haz incidente pueden sufrir absorci\'on fotoel\'ectrica, \textit{scattering}
Compton o producci\'on de pares. De modo que, solo parte de los fotones incidentes alcanzar\'an el detector ubicado detr\'as\footnote{En 
el sentido del haz incidente.} del blanco irradiado. En particular, alcanzar\'an el detector los fotones que no hayan interactuado.
%

%
La probabilidad total por unidad de longitud $d s$ de que un fot\'on incidente no alcance al detector, se denomina coeficiente de 
atenuaci\'on lineal total y representa la integraci\'on de todas las probabilidades correspondientes a cada uno de los posibles procesos de
interacci\'on involucrados.


\section{Procesos para la detecci\'on de neutrones}
% \markboth{Intr. proc. im\'agenes radiol\'ogicas \'ambito m\'edico \ \textbf{M\'ODULO III}}{ESPECIALIDAD III \ \textbf{M\'ODULO III}}
\label{CapIII_2}


La detecci\'on de neutrones presenta algunas caracter\'isticas similares al caso de los fotones, debido a la propiedad de no poseer carga. 
Sin embargo, por su naturaleza intr\'inseca, los procesos involucrados son radicalmente diferentes.
%

%
Los neutrones no interactuan el\'ectricamente con los \'atomos, pero s\'i presentan interacciones fuerte con los n\'ucleos por medio de
una amplia variedad de procesos, entre ellos:

\begin{itemize}
 \item Colisiones el\'asticas, que son relevantes para energ\'ias $\approx \, 1 MeV$,  denominados neutrones r\'apidos).
 \item Colisiones inel\'asticas que son relevantes para valores de energ\'ia superiores al umbral de excitaci\'on nuclear.
 \item Captura de neutrones, proceso por el cual el n\'ucleo captura neutrones incidentes constituyendo un nuevo n\'ucleo, que eventualmente
 puede sufrir transiciones para desexcitarse. Este efecto var\'ia seg\'un la velocidad de los neutrones, aproximadamente inversamente 
 proporcional a \'esta.
 \item Otras reacciones nucleares de tipo $(n,p)$, $(n,d)$, etc que representan captura de un neutr\'on y emisi\'on de part\'iculas cargadas.\,
 Este proceso ocurre en el rango de algunos eV a keV.
 \item Fisi\'on: A energ\'ias ``t\'ermicas'' (del orden del eV), los neutrones se denominan neutrones t\'ermicos o lentos. Este proceso da 
 lugar a la fragmentaci\'on nuclear.
 \item Producci\'on de una \textit{hadronic shower}, efecto que ocurre en el rango de energ\'ias por arriba de unos cientos de keV,
 provocando la emisi\'on de part\'iculas cargadas.
\end{itemize}

Los mecanismos de interacci\'on de los neutrones hacen que su detecci\'on resulte particularmente compleja.
%

%
Sin embargo, existen algunas t\'ecnicas y sistemas de detecci\'on capaces de brindar informaci\'on a cerca del campo de neutrones. 
%
Aunque, el mayor desaf\'io refiere a las dificultades asociadas a determinaciones en campo mixto.

\section{Procesos para la detecci\'on de electrones}
% \markboth{Intr. proc. im\'agenes radiol\'ogicas \'ambito m\'edico \ \textbf{M\'ODULO III}}{ESPECIALIDAD III \ \textbf{M\'ODULO III}}
\label{CapIII_3}


Los electrones y los positrones interactuan por medio de \textit{scattering} con los electrones orbitales at\'omicos con las siguientes 
caracter\'isticas:

\begin{itemize}
 \item Algunos electrones, particularmente los emitidos en las desintegraciones $\beta$, viajan con velocidades relativistas.
 \item Los electrones sufrir\'an cambios significativos en la direcci\'on de movimiento como consecuencia de las colisiones con otros 
 electrones. Por tanto, describen trayectorias err\'aticas (\textit{track}). 
 \item En colisiones frontales con electrones at\'omicos se transfiere una fracci\'on muy importante de la energ\'ia cin\'etica inicial 
 que es adquirida por el electr\'on impactado. Adem\'as, debe destacarse que en estos casos, resulta indistinguible el electr\'on 
 incidente del eyectado.
 \item Debido a cambios abruptos en direcci\'on de movimiento y m\'odulo de la velocidad (energ\'ia cin\'etica), el electr\'on 
 sufre grandes aceleraciones. Como consecuencia, se emite radiaci\'on electromagn\'etica conocida como \textit{Bremsstrahlung}.
\end{itemize}


\section{Procesos para la detecci\'on de part\'iculas cargadas pesadas}
% \markboth{Intr. proc. im\'agenes radiol\'ogicas \'ambito m\'edico \ \textbf{M\'ODULO III}}{ESPECIALIDAD III \ \textbf{M\'ODULO III}}
\label{CapIII_4}

Debido a que los n\'ucleos del material del detector ocupan solamente en torno a 10-15 del volumen de sus \'atomos, resulta unos tres 
\'ordenes m\'as probable para una part\'icula el colisionar con un electr\'on que con un n\'ucleo. Por tanto, el mecanismo de p\'erdida 
de energ\'ia dominante para las part\'iculas cargadas es el \textit{scattering} Coulombiano por los electrones at\'omicos del material 
sensible que compone el detector.
%

%
Si bien el \textit{scattering} Coulombiano de part\'iculas cargadas por los n\'ucleos, denominado \textit{scattering} Rutherford, es un 
proceso importante en f\'isica nuclear, tiene poca influencia en la p\'erdida de energ\'ia de las part\'iculas cargadas a lo largo de su 
trayectoria dentro del detector. 
%

%
Se aplican los principios de conservaci\'on de la energ\'ia y momento en colisiones frontal el\'asticas entre part\'iculas pesadas 
incidentes de masa $M$ y electrones de masa $m_{e}$, supuestos en reposo, para determinar as\'i las probabilidades de los efectos de 
interacci\'on que dan lugar a las secciones eficaces.
%

%
La gran cantidad de eventos de colisi\'on entre part\'icula cargada masiva y electrones del medio material oriogina, entre otras 
consecuencias:

\begin{itemize}
 \item Una gran cantidad de colisiones antes de que la part\'icula ceda toda su energ\'ia cin\'etica. Colisiones frontales generan la 
 m\'axima transferencia posible de energ\'ia. En el resto de las colisiones, la transferencia en general ser\'a mucho menor.
 \item En colisiones entre una part\'icula cargada pesada y un electr\'on, la part\'icula cargada pesada es desviada un \'angulo 
 despreciable, por lo que \'esta sigue una trayectoria pr\'acticamente rectil\'inea.
 \item Dado que la fuerza Coulombiana es de alcance infinito, la part\'icula cargada masiva interactua de modo simultaneo con muchos 
 electrones a la vez, de modo que pierde energ\'ia continua y gradualmente durante la trayectoria. Habiendo recorrido cierta distancia, 
 denominada rango, perder\'a toda la energ\'ia cin\'etica.
\end{itemize}

\section{Detectores gaseosos}
% \markboth{Intr. proc. im\'agenes radiol\'ogicas \'ambito m\'edico \ \textbf{M\'ODULO III}}{ESPECIALIDAD III \ \textbf{M\'ODULO III}}
\label{CapIII_5}


Existen diferentes tipos de sistemas de detcci\'on gaseosos. Esta denominaci\'on proviene del hecho de que el material sensible utilizado 
para la detecci\'on es un gas.


\subsection{C\'amaras de ionizaci\'on}
% \markboth{Intr. proc. im\'agenes radiol\'ogicas \'ambito m\'edico \ \textbf{M\'ODULO III}}{ESPECIALIDAD III \ \textbf{M\'ODULO III}}
\label{CapIII_6}

Los detectores basados en ionizaci\'on est\'an formados esencialmente por un recinto donde se encuentra un gas a presi\'on controlada, all\'i se colocan dos 
electrodos separados una cierta distancia, a los que se aplica una tensi\'on de polarizaci\'on.

El gas dentro del recinto no es conductor el\'ectrico en condiciones normales, por lo tanto no circula corriente el\'ectrica entre los electrodos. Cuando una
part\'icula del haz ionizante interact\'ua con el gas pueden generarse efectos de ionizaci\'on produciendo pares i\'on-electr\'on. El campo el\'ectrico 
someter\'a a las cargas liberadas de modo que se muevan hacia el electrodo de signo contrario; los electrones hacia el \'anodo y los iones hacia el c\'atodo.

La colecci\'on de estas cargas se logra utilizado un dispositivo el\'ectrico asociado al detector, sea midiendo la corriente media que se generada en el 
detector debido a la interacci\'on de varias part\'iculas (c\'amaras que operan en modo corriente) o bien formando un pulso con cada golpe de carga que 
recogen los electrodos (c\'amaras que operan en modo impulso).

Para aplicaciones dosim\'etricas, la c\'amara de ionizaci\'on es un dos\'imetro denominado \textit{standard primario}, ya que adem\'as de ser el sistemas m\'as difundido y utilizado 
universalmente con buena \textit{performance}, sus propiedades permiten obtener mediciones confiables y estables en base a un sistema relativamente simple 
lo que, sumado a teor\'ias s\'olidas respecto de sus principios de funcionamiento, representa una importante ventaja en t\'erminos de estabilidad y 
confiabi-\-
lidad. En este sentido, visto que el funcionamiento del sitema dosim\'etrico est\'a sustentado por teor\'ia de cavidad, como Bragg-Gray, resulta que 
una de las principales caracter\'isticas es el volumen sensible requiere ser determinado de manera particularmente precisa.

En t\'erminos de su uso pr\'actico, la c\'amara de ionizaci\'on se utiliza coloc\'andola expuesta al haz de radiaci\'on o bien introducida en un medio 
material, fantoma, para determinar exposici\'on en aire o bien dosis absorbida en el medio material, t\'ipicamente agua o medios similares en cuanto a las 
propiedades de absorci\'on/dispersi\'on de radiaci\'on ionizante en los rangos de inter\'es. 
Este tipo de medios materiales se denomina ``tejido-equivalentes''. Por tanto, resulta importante tambi\'en conocer las propiedades
del medio material gaseoso en el que se producen los procesos que permiten determinar la dosis absorbida en la cavidad gaseosa.

Existen distintos tipos de c\'amaras de ionizaci\'on. Las m\'as utilizadas son la c\'amara tipo dedal, com\'unmente denominada c\'amara de tipo \textit{Farmer}
y, aunque en menor medida, tambi\'en la c\'amara de ionizaci\'on de tipo plano-paralela.

De hecho, las c\'amaras de ionizaci\'on pueden clasificarse, seg\'un su dise\~no, o m\'as espec\'ificamente seg\'un la forma de los electrodos: 
existen configuraciones planas o
cil\'indri-\-
cas, seg\'un la disposici\'on de los electrodos, que pueden ser  planos-paralelos (c\'amara plano-paralela usualmente denominada Markus), o bien
cil\'indricos, constitu\'idos por un electrodo hueco de forma de cil\'indrica y otro interior en forma de alambre o varilla en dispuesto coaxialmente 
(c\'amara de tipo dedal usualmente llamada Farmer).

El volumen sensible de las c\'amaras de ionizaci\'on se rellenan t\'ipicamente con una variedad de gases que puede ser aire a 
presi\'on atmosf\'erica o bien
gases nobles, especialmente arg\'on.

El rendimiento de detecci\'on, definido como la fracci\'on de de radiaci\'on detectada res-\-
pecto del total que atraviesa el volumen sensible del detector, es
muy pr\'oxima al 100\% para la c\'amara de ionizaci\'on para el caso de la detecci\'on de part\'iculas $\alpha$ (n\'ucleos de helio) y $\beta$ (electrones
y positorones), mientras que para fotones el rendimiento ronda solo el 1\%, debido a las propiedades intr\'insecas de los mecanismos de interacci\'on de
cada tipo de radiaci\'on.

La c\'amara de ionizaci\'on forma parte de una categor\'ia de detectores denominados gaseosos normalmente llamados tambi\'en ``detectores de ionizaci\'on'', 
debido a que este tipo de dispositivos responden a la radiaci\'on por medio de corrientes inducidas por ionizaci\'on.

Adem\'as de la c\'amara de ionizaci\'on, cabe destacar otros dos tipos de detectores gaseosos, his\'orica y a\'un frecuentemente utilizados.

\subsection{Contador proporcional}
% \markboth{Intr. proc. im\'agenes radiol\'ogicas \'ambito m\'edico \ \textbf{M\'ODULO III}}{ESPECIALIDAD III \ \textbf{M\'ODULO III}}
\label{CapIII_7}

En el caso de la c\'amara de ionizaci\'on, el voltage aplicado resulta ser aquel suficiente para colectar solo las cargas liberadas por acci\'on directa de 
la raiaci\'on incidente. Sin embargo, si se aumenta a\'un m\'as el voltaje aplicado, los iones atraidos ganan tanta energ\'ia que podr\'ian generar ionizaciones 
adicionales durante el recorrido hacia los electrodos, y los electrones producidos por estas ionizaciones pueden, a su vez, generar otros, constituyendo un 
efecto en cascada, lo que se conoce como \textit{efecto de amplificaci\'on de la carga por el gas}. El factor por el cual la ionizaci\'on original es 
``multiplicada'' se denomina \textit{factor de amplificaci\'on del gas}. El valor de esta factor aumenta r\'apidamente al aumentar el voltage aplicado y puede 
llegar a valores cercanos a $10^{6}$. Los detectores que operan en este regimen se conocen como contadores proporcionales, y la carga neta puede obtenerse de 
$Q=W*f$, donde $f$ es el factor de amplificaci\'on del gas. Por lo tanto la carga total producida resulta proporcional a la energ\'ia depositada por la 
radiaci\'on ionizante incidente. En general, los contadores proporcionales utilizan gases que permiten la migraci\'on los iones producidos con muy alta 
eficiencia, como los gases nobles, entre lo cuales Ar y Xe son los mas com\'unmente empleados.

\subsection{Contador Geiger-M\"uller}
% \markboth{Intr. proc. im\'agenes radiol\'ogicas \'ambito m\'edico \ \textbf{M\'ODULO III}}{ESPECIALIDAD III \ \textbf{M\'ODULO III}}
\label{CapIII_8}

Los detectores Geiger-M\"uller son detectores gaeosos dise\~nados para obtener la m\'axima amplificaci\'on posible.

El \'anodo central es mantenido a muy alto potencial en relaci\'on al cilindro exterior (c\'atodo). Al producirse ionizaciones dentro de la cavidad de gas por 
interacci\'on de la radiaci\'on incidente, los electrones son acelerados hacia el \'anodo central y los iones positivos al c\'atodo exterior. En este proceso 
ocurre la amplificaci\'on del gas. Pero, debido a que el voltaje aplicado es tan alto, los electrones colectados pueden causar excitaciones de las mol\'eculas 
del gas. Estas mol\'eculas se desexcitan r\'apidamente ($\approx 10^{-9}$s) emitiendo fotones visibles o UV. Si alguno de estos fotones UV interact\'ua con en 
el gas o en el c\'atodo, puede ocurrir fotoabsorci\'on, lo cual genera otro electr\'on para contribuir en el efecto cascada.
%

%
En el caso de los dispositivos de Geiger M\"uller se presenta el problema de que durante la trayectoria de los iones, \'estos pueden ser 
acelerados y alcanzar el \'anodo con la suficiente energ\'ia para liberar electrones y empezar el proceso de nuevo. Esto se debe a 
la naturaleza del proceso de avalancha m\'ultiple en el tubo Geiger, basta con un electr\'on para crear un pulso de salida. Para evitar 
este efecto, se acostumbra a agregar un segundo gas denominado \textit{quenching gas}, o gas de extinci\'on, compuesto por mol\'eculas 
org\'anicas complejas\footnote{El gas de  material sensible, gas primario, es t\'ipicamente aire o un gas noble como arg\'on}.
Se utiliza concentraciones t\'ipicas de 90\% de gas primario y 10\% de gas de extinci\'on.


\section{Detectores de estado l\'iquido y s\'olido}
% \markboth{Intr. proc. im\'agenes radiol\'ogicas \'ambito m\'edico \ \textbf{M\'ODULO III}}{ESPECIALIDAD III \ \textbf{M\'ODULO III}}
\label{CapIII_9}

Estudiados los detectores gaseosos, resulta que presentan algunas desventajas, principalmente asociadas a baja eficiencia para muchos tipos 
de radiaciones, por ejemplo rayos $\gamma$ de 1 MeV, ya que en aire recorre unos 100 m.
%

%
Los detectores de estado s\'olido, que presentan 
densidades mucho mayores, cuentan con la probabilidad de absorci\'on en dimensiones razonables de tama\~no de detecci\'on.
%

%
La principal caracter\'istica de los detectores de estado s\'olido es el uso de matriales s\'olidos para el sensor, es decir material sensible. Desde un
punto de vista general, la utilizaci\'on de materiales sensibles de mayor densidad, prov\'e \textit{a priori} mayor eficiencia en la detecci\'on en cuanto
mayor resulta la cantidad de eventos de interacci\'on, relativamente al caso de materiales gaseosos. Sin embargo, debido a los requerimientos espec\'ificos
para producir efectos secundarios mensurables capacer de ser directa y un\'ivocamente correlacionados con la energ\'ia absotbida por el material, resulta 
que solo algunos pocos materiales de estado s\'olido son \'utiles como material sensible.
%

%
Para crear un detector de estado s\'olido debe estudiarse el compromiso entre:

\begin{enumerate}
 \item El material debe ser capaz de soportar un campo el\'ectrico grande, de manera que los electrones y los iones puedan ser recogidos 
 para formar un pulso electr\'onico. Adem\'as en ausencia de radiaci\'on el flujo de corriente debe ser m\'inimo o nulo para que el ruido 
 de fondo sea bajo. 
 \item Los electrones deben ser f\'acilmente extra\'idos de los \'atomos del material sensible y en gran n\'umero. Electrones e iones deben 
 ser capaces de viajar f\'acilmente en el material.
\end{enumerate}

La primera condici\'on parece exigir un material aislante, mientras que la segunda sugiere usar un conductor. El compromiso, en definitiva, 
es un semiconductor. Materiales semiconductores de tama\~no suficientemente grande para construir detectores de radiaci\'on (de algunas 
decenas de $cm^{3}$) reci\'en estuvieron disponibles a partir de la d\'ecada de 1960.

\subsection{Detectores centelladores}
% \markboth{Intr. proc. im\'agenes radiol\'ogicas \'ambito m\'edico \ \textbf{M\'ODULO III}}{ESPECIALIDAD III \ \textbf{M\'ODULO III}}
\label{CapIII_10}

 Durante la d\'ecada de 1950, debido a la imposibilidad de disponer de materiales semiconductores de dimensiones apropiadas para detecci\'on 
 de radiaci\'on, se desarrollaron los detectores basados en materiales centellantes para aplicaciones en dispositivos de espectroscop\'ia 
 nuclear logrando razonable alta eficiencia resoluci\'on.
 
\begin{center}
\textbf{Detectores semiconductores}
\end{center}

Los detectores semiconductores son, escencialmente, an\'alogos a los detectores gaseosos. Sin embargo, los materiales s\'olidos de los 
semiconductores ofrecen 
importantes ventajas comparativas, ya que cuentan con densidad muy superior a la de los gases\footnote{entre 2 y 5 mil veces mayor, 
aproximadamente. Por ejemplo: 
$\rho_{Si(Li)}=2.33gcm^{-3}$, $\rho_{Ge(Li)}=5.32gcm^{-3}$, $\rho_{Cd(Te)}=6.06gcm^{-3}$ y $\rho_{Aire}=0.001297gcm^{-3}$}. Por lo tanto, 
presenta valores mucho 
mas altos para el \textit{stopping power}, resultando materiales mucho mas eficientes para la detecci\'on de radiaci\'on.
%

%
Los semiconductores son, en general, pobres conductores de corriente el\'ectrica, sin embargo cuando est\'an ionizados por acci\'on de la 
radiaci\'on incidente, 
por ejemplo, la carga el\'ectrica producida puede colectarse por medio de la aplicaci\'on de un voltaje externo. Los materiales m\'as 
comunes para construir 
detectores semiconductores son silicio y germanio, aunque m\'as recientemente se est\'a estableciendo tambi\'en el teluro de cadmio. 
Para estos materiales, una 
ionizaci\'on ocurre cada 3 a 5 eV de energ\'ia absorbida de la radiaci\'on incidente, aproximadamente, lo cual constituye otra importante 
ventaja comparativa 
respecto de los detectores gaseosos. Adem\'as, la amplitd de la se\~nal el\'ectrica detectada est\'a relacionado proporcionalmente con la 
energ\'ia absorbida, y 
por ello pueden ser utilizados para discriminar en energ\'ia.
%

%
Algunas desventajas o inconvenientes de estos dispositivos son: generan corrientes no despreciables a temperatura ambiente, lo cual genera 
un ruido tipo 
\textit{background} en la se\~nal medida, y por tanto deben ser operados a bajas temperaturas. Otro inconveniente es la presencia de 
impurezas en la matriz del 
material, lo cual arruina la configuraci\'on cristalina pura. Estas impurezas crean ``trampas'' electr\'onicas que pueden atrapar electrones 
generados en 
ionizaciones, evitando que sean colectados por los electrodos. Este efecto puede resultar en una apreciable disminuci\'on en la se\~nal 
el\'ectrica medida y 
limita el espesor pr\'actico del material sensible a tama\~nos no mayores a 1cm, aproximadamente. Y, debido al bajo n\'umero at\'omico de 
Si y Ge, este hecho 
limita la posibilidad de emplearlos para detectar fotones, o incluso part\'iculas cargadas, de alta energ\'ia. 
%

%
El paso de la radiaci\'on ionizante a trav\'es de los materiales genera ionizaciones y/o excitaciones. En el caso particular en que las 
especies producidas, 
o residuos,
(ionizadas o excitadas) sufran procesos de recombinaci\'on, se obtiene como resultado la liberaci\'on de energ\'ia. En general, la mayor 
parte de esta energ\'ia 
es disipada en el medio como energ\'ia t\'ermica, por medio de vibraciones moleculares, en el caso de gases y l\'iquidos, o vibraciones de 
red en el caso de 
s\'olidos cristalinos. Sin embargo, existen materiales en los que parte de esta energ\'ia es transferida a emisi\'on de luz visible. Estos 
materiales 
particulares se denominan centelladores y los detectores de radiaci\'on que los utilizan son llamados detectores centelladores.
%

%
Los materiales mas com\'unmente utilizados para detectores de aplicaci\'on en medicina son de tipo org\'anico (substancias org\'anicas 
diluidas en soluci\'on 
l\'iquida) o inorg\'anicos (substancias inorg\'anicas en forma de s\'olido cristalino). Si bien los mecanismos precisos de centelleo son 
diferentes para estos 
dos tipos de materiales, comparten caracter\'isticas comunes. La cantidad de luz producida como consecuencia  de la interacci\'on con un 
\'unico rayo incidente 
(RX, $\gamma$, $\beta$, etc.) resulta proporcional a la energ\'ia depositada por la part\'icula incidente en el material centellador. La 
cantidad de luz neta 
producida es peque\~na, t\'ipicamente unos pocos cientos (a lo sumo miles) de fotones por cada interacci\'on de part\'icula incidiendo 
con energ\'ia de 
entre 70 y 511 keV.
%

%
Originalmente, se utilizaban cuartos oscuros para observar la luz emitida por este tipo de materiales\footnote{Por entonces t\'ipicamente 
centelladores de 
sulfuro.} y contabilizar as\'i la producci\'on de ionizaciones. Esta metodolog\'ia presenta insalvables limitaciones y fue posteriormente 
reemplazada por 
tecnolog\'ias de dispositivos electr\'onicos ultrasensibles a la luz, como los fotomultiplicadores.
%

%
Los detectores por centelleo, generalmente requieren de dispositivos extra para mejorar la eficiencia de lectura. Com\'unmente se utilizan 
t\'ecnicas 
electr\'onicas basadas en arreglo de tubos fotomultiplicadores. 
B\'asicamente, un tubo fotomultiplicador es un dispositivo electr\'onico, en forma de tubo, que produce un pulso  de corriente el\'ectrica 
al ser estimulado 
por se\~nales muy d\'ebiles, como el centelleo producido por RX,  $\gamma$ o $\beta$, etc. en un detector centellador. 
%

%
Se coloca una pel\'icula de material fotoemisor en la ventana de vidrio de entrada, esta sunstancia\footnote{ejemplo t\'ipico es el CsSb 
antomonio de cesio o 
materiales alcalinos.} ejecta electrones cuando son alcanzados por fotones visibles. La superficie de fotoemisi\'on se denomina 
fotoc\'atodo, y los electrones 
ejectados sono fotoelectrones. 
%

%
La eficiencia de conversi\'on de luz visible en electrones liberados se denomina eficiencia cu\'antica, y es t\'ipicamente de entre 1 a 3 
fotoelectrones por 
cada 10 fotones visibles que interact\'uan con el fotoc\'atodo. Claramente, la eficiencia cu\'antica dependende de la energ\'ia de la luz 
incidente.
%

%
Los d\'inodos son mantenidos a diferentes valores de potencial (creciente) para atraer a los electrones producidos, y los secundarios que 
\'estos generan, de modo 
de producir el efecto de multiplicaci\'on. Este proceso se repite usualmente unas 10 veces antes de que la corriente de electrones 
resultante sea colectada por 
el \'anodo. Los factores de multiplicaci\'on que se obtienen son significativos por d\'inodo, resultando en un factor global t\'ipico de 
$10^{7}$, 
aproximadamente. Los tubos fotomultiplicadores se sellan herm\'eticamente y se mantienen en vac\'io; y se construyen en diferentes formas y 
tama\~no.
%

%
Existen tambi\'en detectores de centelleo denominados ``centelladores inorg\'anicos'', ya que consisten en s\'olidos cristalinos que 
centellean debido a 
caracter\'isticas espec\'ificas de la estructura cristalina. Por ello, \'atomos o mol\'eculas individuales de estas substancias no 
centellean, 
se requiere el arreglo cristalino.
%

%
Algunos cristales inorg\'anicos, como el NaI a temperaturas de N l\'iquido, son centella-\-
dores en su estado puro, aunque la mayor\'ia son ``activados con 
impurezas'', y por ello los \'atomos de impurezas\footnote{Indicado como el elemento entre par\'entesis en la notaci\'on del compuesto.} en 
la matriz 
cristalina, responsables del centelleo, se denominan `'centros de activaci\'on''. 

A diferencia del caso de materiales inorg\'anicos, los detectores basados en materia\-
les centelladores org\'anicos, producen el efecto de centelleo debido a 
propiedad inherente de la mol\'ecula de la substancia. El centelleo 
es un mecanismo de excitaci\'on molecular/desexcitaci\'on al interactuar con la radiaci\'on. Este tipo de substancias producen centelleo 
en estado gaseoso, 
l\'iquido o s\'olido, aunque se utilizan normalmente l\'iquidos\footnote{M\'as recientemente han cobrado importancia los centelladores 
pl\'astico}.
En los centelladores org\'anicos l\'iquidos se disuelve el material centellador en un solvente dentro de un contenedor t\'ipicamente de 
vidrio 
o pl\'astico y se agrega tambi\'en la substancia radioactiva a esta mezcla. Se coloca el contenedor entre un par de tubos 
fotomultiplicadores y de este 
modo se detecta la luz emitida que guarda correlaci\'on con la energ\'ia impartida por el material radioactivo.
%

%
Las soluciones de centelladores org\'anicos l\'iquidos consisten de 4 componentes:

\begin{enumerate}
	\item Solvente org\'anico, que compone la mayor parte de la soluci\'on. Debe disolver el material centellador y tambi\'en la 
	muestra radioactiva.
	\item Soluto primario, que absorbe energ\'ia del solvente y emite luz. Algunos materiales centelladores t\'ipicamente utilizados 
	son difenil-oxazol y metilestireno-benceno.
	\item A veces las emisi\'on del soluto primario no es la mas adecuada para ser detectada por los fototubos, y entonces se utiliza 
	un soluto secundario, cuya funci\'on es absorber la emisi\'on del soluto primario y re-emitir fotones, de mayor longitud de onda 
	que los del soluto primario, beneficiando la detectabilidad de la luz por parte de los fototubos.
	\item Frecuentemente se incorporan aditivos a la mezcla para mejorar ciertas propiedades como la eficiencia de tradferencia de 
	energ\'ia. 
\end{enumerate}

El caso particular del detector de NaI(Tl), frecuentemente dise\~nado en forma de campana est\'a formado por el cristal de NaI(Tl) 
ahuecado en un extremo para la inserci\'on 
de la muestra.

La transferencia de luz entre el cristal de NaI(Tl) y el fotomultiplicador resulta ser muy eficiente, aunque existen algunas p\'erdidas 
debido dispersi\'on 
dentro del detector.

La eficiencia de detecci\'on $D$ de un contador de NaI(Tl) en forma de campana para un amplio rango de energ\'ias, principalmente debido 
a 
que la disposici\'on geom\'etrica adoptada implica un eficiencia geom\'etric $g$ muy buena. Entonces, la combinaci\'on con una alta 
eficiencia de detecci\'on 
y un bajo nivel de \textit{background} en el conteo, generan un detector muy eficiente, que puede utilizarse para muestras conteniendo 
cantidades chicas 
(Bq-kBq) de actividad de emisores $\gamma$. La eficiencia geom\'etrica $g$ para muestra de alrededor de 1ml es del 93\%, aproximadamente.

 %
\subsection{Films radiogr\'aficos}
% \markboth{Intr. proc. im\'agenes radiol\'ogicas \'ambito m\'edico \ \textbf{M\'ODULO III}}{ESPECIALIDAD III \ \textbf{M\'ODULO III}}
\label{CapIII_10}

Los films, originalmente radiogr\'aficos, en t\'erminos dosim\'etricos de pobre capacidad, actualmente son reemplazados por films de tipo 
radiocr\'omico que son detectores 
b\'asicamente del tipo qu\'imico. El dise\~no consiste en el dep\'osito de una delgada capa de material sensible sobre una pel\'icula 
inactiva 
t\'ipicamente pl\'astica que proporciona sost\'en. El material sensible consiste en una diluci\'on de sales, usualmente se emplea haluros 
de plata. 
%

%
El principio de funcionamiento se basa en reacciones qu\'imicas catalizadas por la energ\'ia absorbida por el material. Los residuos de 
reacci\'on, que son
substancias con propiedades qu\'imicas diferentes al material sensible en su estado no reactivo, poseen la particular caracter\'istica de 
presentar
afinidad qu\'imica con otros compuestos con los que el material sensible no reactivo no tiene esa afinidad. 
%

%
Se utiliza entonces procesos posteriores a la irradiaci\'on en los que se induce la reacci\'on entre los residuos de formaci\'on a partir 
del material 
sensible -debido a la absorci\'on de energ\'ia- y compuestos con afinidad que una vez unidos producen diferencias en 
absorci\'on/transmisi\'on de luz
visible, es decir presentan diferente opacidad all\'i donde se produce la combinaci\'on entre los productos de reacci\'on por absorci\'on 
de radiaci\'on y
los solutos con afinidad. Este proceso se denomina usualmente revelado radiogr\'afico.
%

%
Una vez realizado el proceso de revelado es necesario implementar un m\'etodo de lectura. Para este fin se aprovecha la manifestaci\'on 
evidente en 
la diferencia de propiedades de absorci\'on/transmisi\'on de luz visible y resulta posible cuantificar estas diferencias empleando 
t\'ecnicas de 
densitometr\'ia \'optica. 
%

%
La respuesta del sistema es, en definitiva, la lectura densitom\'etrica. Y es \'esta la que debe correlacionarse con la dosis absorbida, 
lo cual se realiza
t\'ipicamente mediante m\'etodos emp\'iricos de calibraci\'on. 
%

%
En el caso de los films radiogr\'aficos, la capacidad dosim\'etrica es muy pobre al punto que este tipo de detectores se emplean 
reserv\'andolos casi
exclusivamente para determinaciones espaciales de absorci\'on de radiaci\'on; mientras que los films radiocr\'omicos -m\'as modernos- 
permiten una 
cuantificaci\'on confiable en t\'erminos dosim\'etricos proveyendo tambi\'en informaci\'on espacial.
%

%
Cabe mencionar que la tejiso-equivalencia de los films radiogr\'aficos es muy mala, mientras que esta propiedad mejora para el caso de los 
films
radiocr\'omicos.

A partir de an\'alisis cuantitativos y determinaciones emp\'iricas, resulta que ;la dependencia t\'ipica de la lactura $L$ de un film 
(densidad \'optico) 
presenta una dependencia polinomial (usualmente aproximada por orden 3) respecto de la dosis absorbida. De manera que, en condiciones de 
requerir lineridad
es necesario acotar el intervalo (rango de valores de dosis) y determinar un ajuste lineal para la zona de inter\'es.
%

%
En cualquier caso, ambos tipos de films presentan dificultades en cuanto a la dependencia de la calidad del haz y de la direcci\'on de 
incidencia, aunque
debe destacarse que estos problemas son menosres para el film radiocr\'omico.

 
 
\section{Adaptaci\'on de sistemas de detecci\'on al radiodiagn\'ostico m\'edico}
% \markboth{Intr. proc. im\'agenes radiol\'ogicas \'ambito m\'edico \ \textbf{M\'ODULO III}}{ESPECIALIDAD III \ \textbf{M\'ODULO III}}
\label{CapIII_11}
