\chapter{Procesamiento de im\'agenes con derivadas - Detecci\'on de esquinas y bordes}
% \markboth{Intr. proc. im\'agenes radiol\'ogicas \'ambito m\'edico \ \textbf{M\'ODULO V}}{ESPECIALIDAD III \ \textbf{M\'ODULO V}}
\label{CapV}

El \textit{Cap\'itulo} \ref{CapV} est\'a dedicado a introducir algunos conceptos gen\'ericos, acompa\~nados por m\'etodos espec\'ificos,
para detectar, evidenciar y resaltar puntos de particular importancia en una imagen: puntos de bordes, aristas, contornos.

%

Existe inter\'es y gran practicidad en varios campos de aplicaci\'on, en referencia a la detecci\'on de puntos de bordes, eventualmente la
conexi\'on entre \'estos que dan lugar a contornos, permitiendo as\'i evidenciar la presencia de objetos delimitados quie podr\'an luego 
ser oportunamente caracterizados.

\section{Detecci\'on de bordes utilizando derivadas}
% \markboth{Intr. proc. im\'agenes radiol\'ogicas \'ambito m\'edico \ \textbf{M\'ODULO V}}{ESPECIALIDAD III \ \textbf{M\'ODULO V}}
\label{CapV_1}

Es posible introducir conceptos an\'alogos a la derivaci\'on de c\'alculo continuo de orden 1, 2, etc. a fin de realizar procesamientos
espec\'ificos sobre im\'agenes de inter\'es.

Para regiones con valores de \textit{pixel} constantes, se anular\'a la derivada primera, de modo que puede establecerse precisamente este
criterio para asociar propiedades de la regi\'on con la derivada. 

Por otro lado, si existe cambio en los valores de \textit{pixel}, se generar\'a una variaci\'on en la derivada, permitiendo nuevamente 
establecer una asociaci\'on entre las propiedades de la regi\'on (entorno al punto de inter\'es) y su derivada.

As\'i mismo, se aplica el concepto de derivada segunda, cuyo signo indicar\'a la direcci\'on (positiva o negativa, seg\'un se defina el 
sistema de coordenadas) del gradiente del punto de inter\'es.

\section{Gradiente de una imagen}
% \markboth{Intr. proc. im\'agenes radiol\'ogicas \'ambito m\'edico \ \textbf{M\'ODULO V}}{ESPECIALIDAD III \ \textbf{M\'ODULO V}}
\label{CapV_2}


El operador gradiente $\vec{\nabla}$ aplicado a la imagen $f(m, n)$ se define por medio de:

\begin{eqnarray}
	\vec{\nabla}\left[ f(m, m) \right] \equiv \left[ \begin{array}{c}  \nabla_{m} \left[ f(m, n) \right]  
        \\  \nabla_{n} \left[ f(m, n) \right] \\  \end{array} \right]
	= \left[ \begin{array}{c}  \frac{f(m + \Delta m, n) - f(m - \Delta m, n)}{2 \Delta m} \\ \\  
	 \frac{f(m, n + \Delta n) - f(m, n - \Delta n)}{2 \Delta n} \\  \end{array} \right]
\label{EqLXXIII}
\end{eqnarray}


De modo que la expresi\'on vectorial del gradiente $\vec{\nabla}$ resulta:

\begin{eqnarray}
	\lvert \vec{\nabla}\left[ f(m, m) \right] \rvert = \sqrt{\nabla_{m}^2 + \nabla_{n}^2} \\ \nonumber
	\theta(m, n) = \arctan \left( \frac{\nabla_{m}}{\nabla_{n}} \right)
\label{EqLXXIV}
\end{eqnarray}

N\'otese que, para prop\'ositos de aplicaci\'on dado que se implementar\'an m\'etodos basados en umbralamiento -y por tanto, solo tienen 
relevancia las diferencias relativas- conviene introducir una aproximaci\'on para el m\'odulo del gradiente dada por:

\begin{eqnarray}
	\lvert \vec{\nabla}\left[ f(m, m) \right] \rvert \approx \lvert \vec{\nabla_{m}} \left[ f(m, n) \right]
 \rvert +  \lvert \vec{\nabla_{n}} \left[ f(m, n) \right] \rvert
\label{EqLXXV}
\end{eqnarray}

 
\subsection{Detecci\'on de bordes: El operador de Sobel}
% \markboth{Intr. proc. im\'agenes radiol\'ogicas \'ambito m\'edico \ \textbf{M\'ODULO V}}{ESPECIALIDAD III \ \textbf{M\'ODULO V}}
\label{CapV_3}

Utilizando la expresi\'on \ref{EqLXXIII} es posible practicar una convoluci\'on de la imagen original $f(m, n)$ utilizando una m\'ascara 
$\mathbf{M}$ de $3 \times 3$ siguiendo la definici\'on:


\begin{eqnarray}
	\mathbf{M} \equiv \left[ \begin{array}{ccc}  M_{1, 1} & M_{1, 2} & M_{1, 3}  \\  M_{2, 1} & M_{2, 2} & M_{2, 3} \\ 
	M_{3, 1} & M_{3, 2} & M_{3, 3} \end{array} \right]
\label{EqLXXVI}
\end{eqnarray}

En particular, para el operador de Sobel $M_{Sobel}$ es habitual utilizar las expresiones:

\begin{eqnarray}
	\mathbf{M_{Sobel}} = \begin{array}{c} \left[ \begin{array}{ccc}  -1 & 0 & 1  \\  -2 & 0 & 2 \\ -1 & 0 & 1 \end{array} \right] 
	%\, \, \; \leftrightarrow \nabla_{m}
	\\ %\nonumber \\ \nonumber 
	\left[ \begin{array}{ccc}  -1 & -2 & -1  \\  0 & 0 & 0 \\ 1 & 2 & 1 \end{array} \right] \end{array}
\label{EqLXXVII}
\end{eqnarray}

donde el primer arreglo se utiliza para obtener $\mathbf{\nabla_{m}}$ y el segundo $\mathbf{\nabla_{n}}$. \\

Si el bloque de la imagen original $f(m, n)$, para la dimensi\'on de la 
m\'ascara (3 $\times$ 3) es $B(k, l)$ con 
$k \in [m - \Delta m, m + \Delta m]$ y $k \in [n - \Delta n, n + \Delta n]$

\begin{eqnarray}
	\mathbf{\nabla_{m}} = \left( B_{1, 3} + 2 B_{2, 3} + B_{3, 3} \right) -  
\left( B_{1, 1} + 2 B_{2, 1} + B_{3, 1} \right) 
	\\ \nonumber
	\mathbf{\nabla_{n}} = \left( B_{3, 1} + 2 B_{3, 2} + B_{3, 3} \right)
 -  \left( B_{1, 1} + 2 B_{1, 2} + B_{1, 3} \right) 
\label{EqLXXVIII}
\end{eqnarray}



La aplicaci\'on de la t\'ecnica de detecci\'on y resaltado de bordes, esquinas y contornos por medio del m\'etodo de sobel consiste en 
realizar la convoluci\'on utilizando la expresi\'on \ref{EqLXXVIII}, y luego al resultado aplicar un criterio de umbralamiento por 
medio de un par\'ametro $P_{Sobel}$ de modo que:

\begin{eqnarray}
	\mathbf{M_{Sobel}} \otimes f(m, n) = \left\{ \begin{array}{ll} 
	1 & \mbox{$\lvert \vec{\nabla} \left[ f(m, n) \right] \rvert > P_{Sobel}$}\\
	0 & \mbox{$\lvert \vec{\nabla} \left[ f(m, n) \right] \rvert \le P_{Sobel}$}.\end{array} \right.
\label{EqLXXIX}
\end{eqnarray}


\subsection{Detecci\'on de bordes: El operador de Prewitt}
% \markboth{Intr. proc. im\'agenes radiol\'ogicas \'ambito m\'edico \ \textbf{M\'ODULO V}}{ESPECIALIDAD III \ \textbf{M\'ODULO V}}
\label{CapV_3}

Este operador tambi\'en est\'a definido a partir de la aplicaci\'on de derivadas primeras. De hecho, conceptualmente es similar al operador 
de Sobel, excepto por los valores espec\'ificos de los elementos de matriz de la m\'ascara de convoluci\'on $M_{Prewitt}$, dada por:


\begin{eqnarray}
	\mathbf{M_{Prewitt}} = \begin{array}{c} \left[ \begin{array}{ccc}  -1 & 0 & 1  \\  -1 & 0 & 1 \\ -1 & 0 & 1 \end{array} \right] 
	\\ \nonumber \\ \nonumber 
	\left[ \begin{array}{ccc}  1 & 1 & 1  \\  0 & 0 & 0 \\ -1 & -1 & -1 \end{array} \right] \end{array}
\label{EqLXXX}
\end{eqnarray}

donde el primer arreglo se utiliza para obtener $\mathbf{\nabla_{m}}$ y el segundo $\mathbf{\nabla_{n}}$.


\subsection{Detecci\'on de bordes: El operador de Roberts}
% \markboth{Intr. proc. im\'agenes radiol\'ogicas \'ambito m\'edico \ \textbf{M\'ODULO V}}{ESPECIALIDAD III \ \textbf{M\'ODULO V}}
\label{CapV_4}

La particularidad de este operador es que, diferenci\'andose de los operadores de Sobel y Prewitt, posee la capacidad de evidenciar puntos de
borde pero no as\'i la orientaci\'on de \'estos.

El operador de Roberts se implementa utilizando las diagonales correspondientes al \textit{pixel} de inter\'es, a izquierda $D_{Iz}$ y 
a derecha $D_{De}$, definidas seg\'un:

\begin{eqnarray}
	D_{Iz} \equiv f(m, n) - f(m-1, n-1) \\ \nonumber 
	D_{De} \equiv f(m, n) - f(m-1, n+1)  
\label{EqLXXXI}
\end{eqnarray}

De modo que el operador de Roberts $M_{Roberts}$ consiste en determinar para cada \textit{pixel} la cantidad:

\begin{eqnarray}
	M_{Roberts} \equiv \sqrt{D_{Iz}^2 + D_{De}^2} \, \, \, \rightarrow \, \, 
	M_{Roberts} \approx \lvert D_{Iz} \rvert + \lvert D_{De} \rvert
\label{EqLXXXII}
\end{eqnarray}


\subsection{Detecci\'on de bordes: Operador de Kirsch}
% \markboth{Intr. proc. im\'agenes radiol\'ogicas \'ambito m\'edico \ \textbf{M\'ODULO V}}{ESPECIALIDAD III \ \textbf{M\'ODULO V}}
\label{CapV_5}

El m\'etodo de Kirsch consiste en introducir m\'ascaras $M_{Kirsch}$ que representan 8 rotaciones en las direcciones principales, es decir
4 direcciones en filas y columnas (rotaciones de $0$, $\frac{\pi}{2}$, $\pi$ y $\frac{3 \pi}{2}$), y otras 4 rotaciones respecto de las 
diagonales principales (rotaciones de $\frac{\pi}{4}$, $\frac{3 \pi}{4}$, $\frac{5 \pi}{4}$ y $\frac{7 \pi}{4}$).

Las m\'ascaras de Kirsch se definen como sigue:

\footnotesize

\begin{eqnarray}
	\mathbf{M_{Kirsch}} \equiv \left\{ 
	\begin{array}{cccc} 
	\left[ \begin{array}{ccc}  -3 & -3 & 5  \\  -3 & 0 & 5 \\ -3 & -3 & 5 \end{array} \right] &
	\left[ \begin{array}{ccc}  -3 & 5 & 5  \\  -3 & 0 & 5 \\ -3 & -3 & -3 \end{array} \right] &
	\left[ \begin{array}{ccc}  5 & 5 & 5  \\  -3 & 0 & -3 \\ -3 & -3 & -3 \end{array} \right] &
	\left[ \begin{array}{ccc}  5 & 5 & -3  \\  5 & 0 & -3 \\ -3 & -3 & -3 \end{array} \right]
	\\ 0 & \frac{\pi}{4} & \frac{\pi}{2} & \frac{3 \pi}{4}
	\\ \nonumber
	 \left[ \begin{array}{ccc}  5 & -3 & -3  \\  5 & 0 & -3 \\ 5 & -3 & -3 \end{array} \right] &
	 \left[ \begin{array}{ccc}  -3 & -3 & -3  \\  5 & 0 & -3 \\ 5 & 5 & -3 \end{array} \right] &
	 \left[ \begin{array}{ccc}  -3 & -3 & -3  \\  -3 & 0 & -3 \\ 5 & 5 & 5 \end{array} \right] &
	 \left[ \begin{array}{ccc}  -3 & -3 & -3  \\  -3 & 0 & 5 \\ -3 & 5 & 5 \end{array} \right]
	\\ \pi & \frac{5 \pi}{4} & \frac{3 \pi}{2} & \frac{7 \pi}{4}
	\end{array} \right. 
\label{EqLXXXIII}
\end{eqnarray}

\normalsize 


\subsection{Detecci\'on de bordes: Operadores de Robinson y Frei-Chen}
% \markboth{Intr. proc. im\'agenes radiol\'ogicas \'ambito m\'edico \ \textbf{M\'ODULO V}}{ESPECIALIDAD III \ \textbf{M\'ODULO V}}
\label{CapV_6}

El \textit{set} de m\'ascaras de Robinson es similar al de Kirsch, con la diferencia en los valores de m\'ascara para cada uno 
de los \'angulos.
%
En particular, para \'angulo 0$^{o}$, el operador de m\'ascara de Robinson es:

\begin{eqnarray}
	\mathbf{R_{0}} = \left[ \begin{array}{ccc}  -1 & 0 & 1  \\  -2 & 0 & 2 \\ -1 & 0 & 1 \end{array} \right]
\label{EqAAA}
\end{eqnarray}

El \textit{set} de m\'ascaras de Frei-Chen constituye una autobase, por lo que las 9 componentes del \textit{set} permiten expandir, con 
los pesos correspondientes, cualquier matriz 3 $\times$ 3.
%
Por lo tanto, las m\'ascaras de Fri-Chen, representan una generalizaci\'on de los m\'etodos de \textit{image mask}.
%
Las expresiones de las m\'ascaras de Fri-Chen ($\mathbf{FC_{k}} \; \; k \in [1, 9] $) son:


\begin{eqnarray}
	\mathbf{ FC_{1} \; \; FC_{2} \; \; FC_{3} }  =  \nonumber \\ 
	\frac{1}{2 \sqrt{2}} \left\{ 
	\begin{array}{cccc} 
	\left[  \begin{array}{ccc}  1 & \sqrt{2} & 1  \\  0 & 0 & 0 \\ -1 & -\sqrt{2} & -1 \end{array} \right] &
	\left[ \begin{array}{ccc}  1 & 0 & -1  \\  \sqrt{2} & 0 & -\sqrt{2} \\ 1 & 0 & -1 \end{array} \right] &
	\left[ \begin{array}{ccc}  -1 & 0 & 1  \\  0 & 0 & 0 \\ 1 & 0 & -1 \end{array} \right] &
	\end{array} \right\} \nonumber
\label{EqAAB}
\end{eqnarray}

\begin{eqnarray}
	\mathbf{ FC_{4} \; \; FC_{5} \; \; FC_{6} }  =  \nonumber \\ 
	\frac{1}{2 \sqrt{2}} \left\{ 
	\begin{array}{cccc} 
	\left[  \begin{array}{ccc}  \sqrt{2} & -1 & 0  \\  -1 & 0 & 1 \\ 0 & 1 & -\sqrt{2} \end{array} \right] &
	\left[ \begin{array}{ccc}  0 & 1 & 0  \\  -1 & 0 & -1 \\ 0 & 1 & 0 \end{array} \right] &
	\left[ \begin{array}{ccc}  0 & -1 & \sqrt{2}  \\  1 & 0 & -1 \\ -\sqrt{2} & 1 & 0 \end{array} \right] &
	\end{array} \right\} \nonumber
\label{EqAAC}
\end{eqnarray}

\begin{eqnarray}
	\mathbf{ FC_{7} \; \; FC_{8} }  =  \nonumber \\ 
	\frac{1}{6} \left\{ 
	\begin{array}{cccc} 
	\left[  \begin{array}{ccc}  1 & -2 & 1  \\  -2 & 4 & 2 \\ 1 & -2 & 1 \end{array} \right] &
	\left[ \begin{array}{ccc}  -2 & 1 & -2  \\  1 & 4 & 1 \\ -2 & 1 & -2 \end{array} \right] &
	\end{array} \right\} \nonumber
\label{EqAAD}
\end{eqnarray}

\begin{eqnarray}
	\mathbf{ FC_{9} }  =  \nonumber \\ 
	\frac{1}{3} \left\{ 
	\begin{array}{cccc} 
	\left[  \begin{array}{ccc}  1 & 1 & 1  \\  1 & 1 & 1 \\ 1 & 1 & 1 \end{array} \right] &
	\end{array} \right\}
\label{EqAAE}
\end{eqnarray}

El modo de aplicar la $M$ a la Imagen $I$, que provee el resultado $R$, se obtiene operando del siguiente modo:

\begin{eqnarray}
	R = \sum _{j=1} ^{9} M_{j} \, I_{j} = \lvert M \rvert \; \lvert I \rvert \; \cos(\theta) = V_{M}^{T} \; V_{I} 
\label{EqAAF}
\end{eqnarray}

donde $^{T}$ indica la transpuesta y los vectores $V_{M}$ y $V_{I}$ (de dimensi\'on 1 $\times$ 9) se corresponden con el reordenamiento de 
la matriz en manera vectorial.
%

El \textit{set} de m\'ascaras de Frei-Chen pueden utilizarse para la detecci\'on de bordes, identificando que, en virtud de constituir 
una autobase, un subespacio arbitrario se asocia con ``proyecciones'' de la imagen (o bloque) en el subespacio de inter\'es.



\section{Extensi\'on de los operadores}
% \markboth{Intr. proc. im\'agenes radiol\'ogicas \'ambito m\'edico \ \textbf{M\'ODULO V}}{ESPECIALIDAD III \ \textbf{M\'ODULO V}}
\label{CapV_7}


Puede verse que uno de los problemas usuales en los m\'etodos de detecci\'on de bordes, como en general en cualquier t\'ecnica de 
procesamiento, es que la presencia de se\~nal esp\'urea como ruido, perjudica y limita significativamente la capacidad y 
\textit{performance} de los m\'etodos de procesamiento.
%

Un intento para reducir los efectos limitantes de la presencia de ruido se basa en el concepto de ``extensi\'on de operadores'', que 
consiste en implementar un procedimiento previo a aplicar operadores de detecci\'on de bordes con el objeto conseguir una reducci\'on 
del ruido en la se\~nal.
%
Pero, en lugar de realizar procesos t\;ipicos de filtrado de ruido, como alguno de los descritos en el Cap\'itulo \ref{CapII}, se propone
una ``expansi\'on'' de los operadores de detecci\'on de bordes, que consiste b\'asicamente en aumentar la dimensi\'on de la m\'ascara 
correspondiente sin alterar las propiedades de simetr\'ia y operaci\'on de dicha m\'ascara.
%

A modo de ejemplo, la extensi\'on o expansi\'on del operador de Sobel ser\'ia:

\begin{eqnarray}
	%\begin{array}{ccccccc} 
	\left[  \begin{array}{ccccccc}  -1 & -1 & -1  & -2 & -1 & -1 & -1\\  -1 & -1 & -1  & -2 & -1 & -1 & -1 \\ 
	-1 & -1 & -1  & -2 & -1 & -1 & -1 \\ 0 & 0 & 0  & 0 & 0 & 0 & 0 \\ 1 & 1 & 1  & 2 & 1 & 1 & 1 \\
	1 & 1 & 1  & 2 & 1 & 1 & 1 \\ 1 & 1 & 1  & 2 & 1 & 1 & 1 
	\end{array} \right] %&
	%\end{array} 	\right\}
\label{EqAAG}
\end{eqnarray}


El c\'alculo del gradiente por fila $G_{X}$ o por columna $G_{y}$ se obtienen por correspondencia de rotaciones $\frac{\pi}{2}$ del 
operador, en el ejemplo el de Sobel.
%

La extensi\'on de los operadores, originalmente de 3 $\times$ 3 puede realizarse hacia cualquiera dimensi\'on, aunque t\'ipicamente se
realiza para 11 $\times$ 11, 9 $\times$ 9 y 7 $\times$ 7.

\section{El m\'etodo de Canny: Algoritmo}
% \markboth{Intr. proc. im\'agenes radiol\'ogicas \'ambito m\'edico \ \textbf{M\'ODULO V}}{ESPECIALIDAD III \ \textbf{M\'ODULO V}}
\label{CapV_8}

El operador de detecci\'on de bordes de Canny desarrollado durante los '80 se basa en un algoritmo de m\'ultiples fases capaz de detectar 
bordes de diferentes caracter\'isticas. 
%
Representa,de hecho, el operador m\'as utilizado en la detecci\'on de bordes.
%

El objetivo ideal del m\'etodo propuesto por Canny consist\'ia en hallar un algoritmo \'optimo para detectar bordes.\footnote{V\'ease, por 
ejemplo \textit{Mammography image detection processing for automatic micro-calcification recognition} Quintana et al. y 
\textit{Mammography image quality optimisation: a Monte Carlo study} Tirao et al. donde se encuentran ejemplos del c\'alculo y uso de 
operadores de Canny por medio de gradientes por fila y columna.} 
%
El concepto b\'asico es que un buen mecanismo de detecci\'on \'optimo es aquel algoritmo capaz de marcar tantos bordes reales como sea 
posible, una buena localizaci\'on, y los bordes marcados deben estar lo m\'as cerca posible del borde en la imagen real. 
%
Adem\'as, procurar que un borde dado debe ser marcado s\'olo una vez y donde sea posible el ruido presente en la imagen no deber\'ia crear 
falsos bordes.
%

La implementaci\'on del m\'etodo de Canny permite optimizar la detecci\'on de bordes diferenciales y consta de tres etapas principales: 
filtrado, decisi\'on inicial, e hist\'eresis.
%

La primera etapa consiste en un filtrado de convoluci\'on de la derivada primera de una funci\'on Gaussiana normalizada discreta sobre la 
imagen, realizada en dos direcciones: horizontal y vertical. 
%
La funci\'on Gaussiana posee dos par\'ametros fundamentales, valor medio $M$ y desviaci\'on t\'ipica $\sigma$.
%
En este caso, el valor medio es nulo, por lo tanto la ecuaci\'on que define el filtro Gaussiano $G(x)$ es:

\begin{eqnarray}
	G(x) = k \, \: e^\frac{-x^2}{2 \, \sigma^2}
\label{EqAA}
\end{eqnarray}

donde el par\'ametro $k$ se determina de manera que el m\'aximo de $G(x)$ sea la unidad en su versi\'on discreta. 
%

Para la realizaci\'on del filtro utilizado en la primera etapa, se deriva la expresi\'on \ref{EqAA}, obteni\'endose:

\begin{eqnarray}
	\frac{d \, G(x)}{d \, x} = -\frac{k}{\sigma^2} \, \: x \, \; e^\frac{-x^2}{2 \, \sigma^2}
\label{EqAB}
\end{eqnarray}

La versi\'on discreta del filtro se define de modo an\'alogo y se considera que $x$ asume valores en $[-5 \sigma, 5 \sigma]$ con 
diferencias de 1 pixel entre muestras consecutivas. 
%

obviamente, por razones de eficiencia de c\'omputo, es preferible derivar directamente la versi\'on discreta de \ref{EqAA}, con el fin de 
obtener el valor de $k$.