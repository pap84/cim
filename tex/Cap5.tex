\chapter{Procesos estoc\'asticos}
% \markboth{Intr. proc. im\'agenes radiol\'ogicas \'ambito m\'edico \ \textbf{M\'ODULO VI}}{ESPECIALIDAD III \ \textbf{M\'ODULO VI}}
\label{CapVI}

El \textit{Cap\'itulo} \ref{CapVI} est\'a dedicado a los elementos b\'asicos sobre procesos estoc\'asticos. Se introducen conceptos generales, 
consideraciones estad\'isticas y fen\'omenos f\'isicos de carc\'acter intr\'insecamente aleatorio. Se dedica especial atenci\'on al 
transporte de radiaci\'on en su caracter\'istica estoc\'astica.

\section{Introducci\'on y definiciones de procesos estoc\'asticos}
% \markboth{Intr. proc. im\'agenes radiol\'ogicas \'ambito m\'edico \ \textbf{M\'ODULO V}}{ESPECIALIDAD III \ \textbf{M\'ODULO V}}
\label{CapVI_1}

En los procesos estoc\'asticos se representan \underline{todos y cada uno} de los pasos necesarios para la realizaci\'on de un cieto evento 
as\'i como tambi\'en los maneras en que cada uno de los pasos puede ser realizado en t\'erminos de las respectivas probabilidades. 
%
Por tanto, cualquier proceso en el que se vean involucradas probabilidades de ocurrencia resulta ser un proceso estoc\'astico.

Al describir variables de car\'acter aleatorio, vinculadas a fen\'omenos de tipo probabil\'isticos como lo es el transporte de radiaci\'on, 
es asumido, como premisa impl\'icita por defecto, el hecho de que las caracter\'isticas aleatorias permanecen constantes durante el 
intervalo de tiempo de inter\'es, aunque desde una perspectiva gen\'erica podr\'ia no satisfacerse esta asumpci\'on.

En efecto, al incorporar la dependencia (o evoluci\'on) de variables consideradas determin\'isticas, \'estas describir\'an un proceso 
evolutivo de tipo anal\'itico, mientras que para el caso de variables aleatorias mostrar\'an una evoluci\'on condicionada por el v\'inculo
al fen\'omeno probabil\'istico asociado.
%
Entonces, toda funci\'on definida a partir de variables aleatorias, como por ejemplo funciones de distribuci\'on o funciones de densidad, 
presentar\'an dependencia temporal determinada por su car\'acter aleatorio, dando lugar a la naturaleza estoc\'astica del fen\'omeno f\'isico 
involucrado.

Una definici\'on m\'as formal de un proceso estoc\'astico es la siguiente:


\textsl{ El proceso estoc\'astico consiste en el conjunto (o familia) de variables aleatorias $\{X_{t} t \in [t_{ini}, t_{fin}]\}$ que se
ordenan de acuerdo con el \'indice $t$, por lo general identificando al tiempo.}

En consecuencia, se tiene que para cada valor de $t$ (instante) existe la variable aleatoria representada por $X_{t}$, de modo que el 
proceso estoc\'astico puede interpretarse como una sucesi\'on de variables aleatorias, las que pueden variar (evolucionar) en sus 
caracter\'isticas.

Los \textsl{estados de variables aleatorias} son los posibles valores que \'estas pueden asumir.
%
Por lo tanto, existe un \textsl{espacio de estados} asociados a las variables aleatorias.
%
En particular, la variable temporal $t$ puede ser de tipo discreto o bien de tipo continuo. La modificaci\'on de la variable $t$, por 
ejemplo, dar\'ia lugar a cambios de estado que ocurren en el instante $t$.

Por tanto, de acuerdo con el conjunto de \'indices\footnote{Estrictamente, sub\'indices.} $t \in T=[t_{ini}, t_{fin}]$, la variable 
aleatoria $X_{t}$ puede clasificarse seg\'un los siguientes criterios para procesos estoc\'asticos:

\begin{itemize}
 \item Si el conjunto $T$ es continuo (por ejemplo $\Re^{+}$), resulta que $X_{t}$ describe un proceso estoc\'astico de par\'ametro continuo.
 %
 \item Si el conjunto $T$ es dicreto, $X_{t}$ describe un proceso estoc\'astico de par\'ametro discreto.
 %
 \item Si para cada valor (instante) $t$ la variable aleatoria $X_{t}$ es de tipo continuo, resulta que proceso estoc\'astico es de estado 
 continuo.
 %
 \item Si para cada valor (instante) $t$ la variable aleatoria $X_{t}$ es de tipo discreto, resulta que proceso estoc\'astico es de estado 
 discreto.
\end{itemize}

Una \textsl{cadena} es un proceso estoc\'astico para el cual el tiempo evoluciona de manera discreta y la variable aleatoria s\'olo puede 
tomar valores discretos en el espacio de estados correspondiente. 
%

Un \textsl{proceso de saltos puros} es un proceso estoc\'astico para el cual los cambios de estados suceden de forma aislada y aleatoria 
pero la variable aleatoria s\'olo asume valores discretos en el espacio de estados correspondiente. 
%
Diversamente, un \textsl{proceso continuo} se refiere al caso en que los cambios de estado se producen para cualquier valor de $t$ (instante)
y hacia cualquier estado dentro de un espacio continuo de estados correspondiente.


\subsection{Procesos de estado discreto y cadenas de Markov}
% \markboth{Intr. proc. im\'agenes radiol\'ogicas \'ambito m\'edico \ \textbf{M\'ODULO V}}{ESPECIALIDAD III \ \textbf{M\'ODULO V}}
\label{CapVI_2}

En el caso de procesos estoc\'asticos con espacio de estados discreto, una secuencia de variables que indique el valor del proceso en 
instantes sucesivos\footnote{Se asume que la variable $t$ refiere al tiempo.} puede representarse del siguiente modo:


\begin{eqnarray}
	\{ X_{0} = x_{0}, X_{1} = x_{1}, ... , X_{n} = x_{n} \}
\label{EqLXXXIV}
\end{eqnarray}

donde cada variable $X_{j} \, \: j \in [1, n]$ presenta una distribuci\'on de probabilidades tal que, en general, es diferente de las otras 
variables aunque podr\'ia haber caracter\'isticas comunes.

Uno de los principales objetivos del estudio del caso discreto es el c\'alculo de proba-\-
bilidades 
de ocupaci\'on de cada estado a partir de las probabilidades de cambio de estado. 
%
Si para el valor $t_{j-1}$ (instante) el sistema est\'a en el estado $x_{j-1}$, la probabilidad 
de que al 
instante siguiente $t_{j}$ se encuentre en el estado $x_{j}$ se obtiene a partir de la 
probabilidad de transici\'on o cambio de estado de $x_{j-1}$ a $x_{j}$ (o probabilidad 
condicionada) denotada por $P\left( X_{j} = x_{j} / X_{j-1} = x_{j-1} \right) = P_{j, j-1}$, donde 
$P_{j, j-1}$ es el valor que asume la probabilidad para el caso espec\'ifico en consideraci\'on.
%

Las probabilidades del tipo $P \left( X_{j} = x_{j} \right)$ se denominan probabilidades de 
ocupaci\'on de estado.
%

De modo similar, otro tipo de probabilidad de inter\'es es la de ocupar un cierto estado en un 
instante $t_{j}$, dado que en todos los 
instantes anteriores, desde $t_{ini}$ a $t_{j-1}$ se conoce en qu\'e estados estuvo el proceso. 
%
En este caso, la probabilidad condicionada es 
$P \left( X_{j} = x_{j} / X_{ini} = x_{ini}, \, ... , \, X_{j-1} = x_{j-1} \right) = 
P_{ini, ..., j-1, j}$

Por tanto, la probabilidad $P_{ini, ..., j-1, j}$ depende de toda la ``historia pasada del 
proceso'', mientras que la probabilidad de transici\'on depende \'unicamente del estado actual que 
ocupe el proceso.
%\vspace{0.25cm}
\newpage

\textbf{Propiedad de Markov:}

Se dice que un proceso cumple la propiedad de Markov cuando toda la historia pasada del proceso se puede resumir en la posici\'on actual 
que ocupa el proceso para poder calcular la probabilidad de cambiar a otro estado. Es decir, se cumple:

\begin{eqnarray}
	P \left( X_{j} = x_{j} / X_{ini} = x_{ini}, \, ... , \, X_{j-1} = x_{j-1} \right) =  
	P \left( X_{j} = x_{j} /  X_{j-1} = x_{j-1} \right)
\label{EqLXXXVII}
\end{eqnarray}


Adem\'as, una propiedad importante que puede tener una cadena es que los valores $p_{mn} (j)$ no dependan del valor de $j$.
%
Entonces, se tiene que las probabilidades de cambiar de estado son las mismas en cualquier instante. 
%
Por lo tanto, esta propiedad indica que las probabilidades de transici\'on son estacionarias.


\subsection{Procesos de saltos puros}
% \markboth{Intr. proc. im\'agenes radiol\'ogicas \'ambito m\'edico \ \textbf{M\'ODULO V}}{ESPECIALIDAD III \ \textbf{M\'ODULO V}}
\label{CapVI_3}

En este caso, el proceso sigue siendo discreto en estados pero la gran diferencia es que los cambios de estado ocurren en cualquier 
instante en el tiempo (tiempo continuo). 
%

Un proceso estoc\'astico en tiempo continuo $\{ N(t) \, t \ge 0 \}$ se denomina \textsl{proceso de conteo} si representa el n\'umero de 
veces que ocurre un suceso hasta el instante de tiempo $t$.
%

En particular, se tiene $N(t) \in \mathbf{N}$ y $N(t^*) \le N(t) \, \; \forall t^* < t$.
%

Un proceso de conteo es un \textsl{proceso de Poisson homog\'eneo} de tasa $\lambda$ si satisface:

\begin{enumerate}
 \item $N(0) = 0$
 \item $N(t_{k}) - N(t_{k-1})$ es una variable aleatoria independiente (proceso de incrementos independientes) $\forall \, k$.
 \item $N(t + t^*) - N(t^*)$, que denota la cantidad de eventos que ocurren entre el instante $t^*$ y $t$, sigue una distribuci\'on de 
 Poisson de par\'ametro $\lambda t$.
\end{enumerate}


\subsection{Procesos de estados continuos y series temporales}
% \markboth{Intr. proc. im\'agenes radiol\'ogicas \'ambito m\'edico \ \textbf{M\'ODULO V}}{ESPECIALIDAD III \ \textbf{M\'ODULO V}}
\label{CapVI_4}

Un concepto importante en procesos estoc\'asticos es la \textsl{realizaci\'on}, o bien una realizaci\'on de una experiencia aleatoria, 
que es el resultado de una repetici\'on de esa experiencia. 
%
Por tanto, en la experiencia aleatoria de ``lanzar una vez un dado'' una realizaci\'on posible ser\'ia obtener el n\'umero 2, en el \'unico 
lanzamiento hecho. 
%
En ese caso, la realizaci\'on se reduce a un \'unico n\'umero $\{X\}$. 
%
Si se repite la experiencia, podr\'ian obtener otras realizaciones (cualquiera de los n\'umeros 1, 3, 4, 5 y 6).
%

En una experiencia $M$-dimensional, una realizaci\'on es el resultado obtenido de los $M$ par\'ametros, denotado por $\{X_{1}, ..., X_{M} \}$.
%

Una \textsl{serie temporal} es una realizaci\'on parcial de un proceso estoc\'astico de par\'ametro tiempo discreto.
%
De aqu\'i que la teor\'ia de los procesos estoc\'asticos es de aplicaci\'on a las series temporales.
%
Sin embargo, existe una fuerte restricci\'on que radica en el hecho de que en muchas series temporales, ellas son la \'unica realizaci\'on 
observable del proceso estoc\'astico asociado.
%

\section{Caracter\'isticas y medidas de procesos estoc\'asticos}
% \markboth{Intr. proc. im\'agenes radiol\'ogicas \'ambito m\'edico \ \textbf{M\'ODULO V}}{ESPECIALIDAD III \ \textbf{M\'ODULO V}}
\label{CapVI_5}

Para un espacio de estados $M$-dimensional, pueden calcularse cantidades y medidas estad\'isticamente representativas para los estados 
descritos por las variables $M$-dimensionales.
%
En particular, se definen -entre tantos- medidas como tensores de valor medio y de covarianzas, que permiten obtener caracter\'isticas 
representativas de los procesos estoc\'astico.
%


\begin{center}
{\bf {blue}{Cap\'itulo 1 Manual PENELOPE v. 2008}}
\end{center}



\section{Procesos estoc\'asticos estacionarios}
% \markboth{Intr. proc. im\'agenes radiol\'ogicas \'ambito m\'edico \ \textbf{M\'ODULO V}}{ESPECIALIDAD III \ \textbf{M\'ODULO V}}
\label{CapVI_6}

En primera aproximaci\'on, se considerar\'an estacionarios a los procesos estoc\'asticos que tengan un comportamiento constante a lo 
largo del tiempo. 
%

Un \textsl{proceso estoc\'astico estacionario en sentido estricto} requiere que al realizar un mismo desplazamiento en el tiempo de todas 
las variables de cualquier distribuci\'on conjunta finita se obtenga que esta distribuci\'on no var\'ia.
%
Es decir:

\begin{eqnarray}
	F \left( X_{i_1}, ... , X_{i_M} \right) =   F \left( X_{i_1 + j}, ... , X_{i_M + j} \right) \: \, \forall i_k , \, j
\label{EqLXXXVIII}
\end{eqnarray}


En cambio, un \textsl{proceso estoc\'astico esestacionario en sentido d\'ebil} requiere que se mantengan constantes todas sus 
caracter\'isticas lo largo del tiempo.
%
Es decir, que $\forall t$:

\begin{enumerate}
 \item $\langle X_t \rangle = \langle X \rangle \; \, \forall t$ donde $\langle X \rangle$ denota el valor medio o de expectaci\'on.
 \item $\sigma_{X_t}  = \sigma_{X} \; \, \forall t$ donde $\sigma_{X}$ denota la varianza.
 \item $Cov \left( t, t+j \right) = Cov \left( t^*, t^*+j \right) = C_{j} \, \; \forall j = 0, \pm 1, \pm 2, ...$ donde $Cov$ denota la 
 covarianza y $C$ es una constante.
\end{enumerate}


\subsection{Procesos de ruido blanco}
% \markboth{Intr. proc. im\'agenes radiol\'ogicas \'ambito m\'edico \ \textbf{M\'ODULO V}}{ESPECIALIDAD III \ \textbf{M\'ODULO V}}
\label{CapVI_7}

Un proceso estoc\'astico utilizado frecuentemente es el de ``ruido blanco'', dado por el proceso estacionario $RB_{t}$ que satisface:

\begin{itemize}
 \item $\langle RB_{t} \rangle = \langle RB \rangle = 0  \, \: \forall t$
 \item $ Var(RB_{t}) = \sigma^2$
 \item $ Cov(RB_{t}, RB_{t^*}) = 0 \; \, t^* \ne t$
\end{itemize}

En este sentido, puede interpretarse al ruido blanco como una sucesi\'on de valores sin relaci\'on alguna entre ellos, oscilando en torno 
al cero dentro de un margen constante. 
%
En este tipo de procesos, conocer valores pasados no proporciona ninguna informaci\'on sobre el futuro ya que el proceso es ``puramente 
aleatorio'', y por consiguiente ``carece de memoria''.

\section{El transporte de radiaci\'on como proceso estoc\'astico}
% \markboth{Intr. proc. im\'agenes radiol\'ogicas \'ambito m\'edico \ \textbf{M\'ODULO V}}{ESPECIALIDAD III \ \textbf{M\'ODULO V}}
\label{CapVI_8}

\begin{center}
{\bf {blue}{Introducci\'on via secciones eficaces explicando analog\'ia pdf-DCS y explicar pag 6-15 Manual PENELOPE v. 2008}}
\end{center}


\section{Reformulaci\'on integral de la ecuaci\'on de transporte}
% \markboth{Intr. proc. im\'agenes radiol\'ogicas \'ambito m\'edico \ \textbf{M\'ODULO V}}{ESPECIALIDAD III \ \textbf{M\'ODULO V}}
\label{CapVI_9}

A partir de la expresi\'on \'integro-diferencial de la ecuaci\'on de transporte de Boltzmann (\ref{EqX}), es posible reformular los 
t\'erminos para arribar a una ecuaci\'on completamente integral, lo cual resulta de particular utilidad para el manejo de soluciones 
de tipo num\'ericas, necesarias para situaciones realistas, ya que -como se sabe- las soluciones anal\'iticas directas s\'olo son 
posibles en una cantidad miuy limitada de configuraciones.
%

Operando y reordenando los t\'erminos en la ecuaci\'on de Boltzmann \ref{EqX}, resulta:


\begin{eqnarray}
 t = t_{0} + \frac{s}{\lvert\vec{v}\rvert}       \nonumber \\
 \vec{r} = \vec{r_{0}} + s\, \vec{\Omega}
 \label{EcXI}
\end{eqnarray}

Por lo tanto, se obtiene:

\begin{eqnarray}
	\frac{d}{d s} \, \Psi \left( \vec{r_{0}} + s \vec{\Omega}, \vec{\Omega}, E, t_{0} +\frac{s}{\lvert\vec{v}\rvert}  \right) + 
	\Sigma \; \Psi \left( \vec{r_{0}} + s \vec{\Omega}, \vec{\Omega}, E, t_{0} +\frac{s}{\lvert\vec{v}\rvert} \right)
	= \\
	\Gamma \left( \vec{r_{0}} + s \vec{\Omega}, \vec{\Omega}, E, t_{0} +\frac{s}{\lvert\vec{v}\rvert} \right)
 \label{EqXII}
\end{eqnarray}

donde se ha definido $\Gamma \left( \vec{r_{0}} + s \vec{\Omega}, \vec{\Omega}, E, t_{0} +\frac{s}{\lvert\vec{v}\rvert} \right)$ como sigue:

\begin{equation}
 \Gamma \equiv S + \iint \, \Sigma _{s} \left( \vec{r_{0}} + s \vec{\Omega}, (\vec{\Omega'}, E') \rightarrow (\vec{\Omega}, E)  \right)
	\Psi \left( \vec{r_{0}} + s \vec{\Omega}, \vec{\Omega'}, E', t_{0} +\frac{s}{\lvert\vec{v}\rvert}  \right) \, \, d \, \vec{\Omega'} \, d \, E'
 \label{EqXIII} 
\end{equation}

Puede verse\footnote{\underline{Hint:} Introd\'uzcase $e^{ \int _{-\infty} ^{s} \, \, \Sigma \left( \vec{r_{0}} + s' \vec{\Omega}, E \right) \, \, ds'}$
y calc\'ulese $\frac{d}{d s} \Psi $ .}

\begin{equation}
 	\Psi \left( \vec{r_{0}}, \vec{\Omega}, E, t_{0} \right) = \int  _{-\infty} ^{0} \, \: ds \left[ e^{ \int _{0} ^{s} 
 	\Sigma \left( \vec{r_{0}} - s' \vec{\Omega}, E \right) \, ds' }  \; \: 
 	\Gamma \left( \vec{r_{0}} + s \vec{\Omega}, \vec{\Omega}, E, t_{0} +\frac{s}{\lvert\vec{v}\rvert} \right) \right]
 \label{EqXIV} 
\end{equation}


Considerando que las variables $\vec{r_{0}}$ y $t_{0}$ son arbitrarias, se obtiene:

\begin{eqnarray}
 	\Psi \left( \vec{r}, \vec{\Omega}, E, t \right) =   %\nonumber \\
 	 \int  _{0} ^{\infty} \, \: e^{ \int _{0} ^{s} \Sigma \left( \vec{r_{0}} - s' \vec{\Omega}, E \right) \, ds' }  \; \cdotp \; \: \nonumber \\
 	\left[ \iint \Sigma_{s} \left( \vec{r} - s \vec{\Omega}, (\vec{\Omega'}, E') \rightarrow (\vec{\Omega}, E)  \right)
 	\Psi \left( \vec{r} - s \vec{\Omega}, \vec{\Omega}, E, t - \frac{s}{\lvert\vec{v}\rvert} \right) %\nonumber \\
 	+  S \left( \vec{r} - s' \vec{\Omega}, \vec{\Omega}, E, t  \right)  \right]
        %\right]
 \label{EqXV} 
\end{eqnarray}

Es decir, se obtuvo una forma integral para la ecuaci\'on de Boltzmann, que puede escribirse en t\'ermino de operadores\footnote{Resulta conveniente
expresar la ecuaci\'on de este modo para la resoluci\'on num\'erica de la misma, por ejemplo utilizando m\'etodos estad\'isticos como Monte Carlo.}:


\begin{equation}
 	\Psi = \mathbf{K} \; \Psi + S'
 \label{EqXVI} 
\end{equation}

Se obtiene la soluci\'on para el flujo:

\begin{equation}
 	\Psi = \Sigma _{i=0} ^{\infty} \Psi_{i}
 \label{EqXVII} 
\end{equation}

Donde los t\'erminos son:

\begin{eqnarray}
 	\Psi_{i} = \mathbf{K} \; \Psi_{i-1} \nonumber \\
 	\Psi_{0} = S'
 \label{EqXVIII} 
\end{eqnarray}

Matem\'aticamente, la soluci\'on obtenida se denomina serie de von Neuman. La interpretaci\'on f\'isica del formalismo desarrollado es particularmente
apropiada en el v\'inculo entre los t\'erminos de la serie y los procesos f\'isicos involucrados. El t\'ermino de orden 0 se refiere al flujo primario
estrictamente proveniente de la fuente de emisi\'on $S$, mientras que los t\'erminos $\Psi_{i}$ son las contribuciones de \textit{scattering} a orden $i$
obtenidas a partir del operador del \textit{kernel de scattering} $\mathbf{K}$.