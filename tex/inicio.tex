\booktitle{Introducci\'on al procesamiento de im\'agenes radiol\'ogicas en el \'ambito cl\'inico}
\subtitle{Bases teóricas y analíticas}

\authors{Mauro Valente \\
% \affil{}
Pedro P\'erez
% \affil{Facultad de Matemática, Astronomía, Física y Computación}
}

\offprintinfo{Introducci\'on al procesamiento de im\'agenes radiol\'ogicas en el \'ambito cl\'inico}{Valente y Pérez}

%% Can use \\ if title, and edition are too wide, ie,
%% \offprintinfo{Survey Methodology,\\ Second Edition}{Robert M. Groves}

%%%%%%%%%%%%%%%%%%%%%%%%%%%%%%
%% 
% \halftitlepage

\titlepage


\begin{copyrightpage}{2016}
Introducci\'on al procesamiento de im\'agenes radiol\'ogicas en el \'ambito cl\'inico / M. Valente y P. Pérez.
\    Incluye referencias bibliográficas e índice.
\    ISBN (no publicado)
% \    1. Surveys---Methodology.  2. Social 
% \  sciences---Research---Statistical methods.  I. Groves, Robert M.  II. %
% Series.\\

% HA31.2.S873 2007
% 001.4'33---dc22                                             2004044064
\end{copyrightpage}



\dedication{...}

\begin{contributors}
\name{Colaborador N. U.,} Trabaja acá, En esta ciudad, En este país

\name{Colaborador N. D.,} Trabaja acá, En esta ciudad, En este país
\end{contributors}

% \contentsinbrief
\tableofcontents
% \listoffigures
% \listoftables


% \begin{foreword}
% This is the foreword to the book.
% \end{foreword}
% 
% \begin{preface}
% This is an example preface.
% This is an example preface.
% This is an example preface.
% This is an example preface.
% 
% \prefaceauthor{R. K. Watts}
% \where{Durham, North Carolina\\
% September, 2007}
% 
% \end{preface}
% 
% 
% \begin{acknowledgments}
% From Dr.~Jay Young, consultant from Silver Spring, Maryland, I received
% the initial push to even consider writing this book. Jay was a constant
% ``peer reader'' and very welcome advisor durying this year-long process.
% 
% 
% To all these wonderful people I owe a deep sense of gratitude especially now
% that this project has been completed.
% \authorinitials{G. T. S.}
% \end{acknowledgments}
% 
% \begin{acronyms}
% \acro{ACGIH}{American Conference of Governmental Industrial Hygienists}
% \acro{AEC}{Atomic Energy Commission}
% \acro{OSHA}{Occupational Health and Safety Commission}
% \acro{SAMA}{Scientific Apparatus Makers Association}
% \end{acronyms}
% 
% \begin{glossary}
% \term{NormGibbs}Draw a sample from a posterior distribution
% of data with an unknown mean and variance using Gibbs sampling.
% 
% \term{pNull}Test a one sided hypothesis from a numberically
% specified posterior CDF or from a sample from the posterior
% 
% \term{sintegral}A numerical integration using Simpson's rule
% \end{glossary}
% 
% \begin{symbols}
% \term{A}Amplitude
% 
% \term{\hbox{\&}}Propositional logic symbol 
% 
% \term{a}Filter Coefficient
% 
% \bigskip
% 
% \term{\mathcal{B}}Number of Beats
% \end{symbols}

% \begin{introduction}

%% optional, but if you want to list author:

% \introauthor{Catherine Clark, PhD.}
% {Harvard School of Public Health\\
% Boston, MA, USA}
% 
% The era of modern \index{microelectronics}\index{microelectronics!modern} 
% began in 1958 with the invention of the
% integrated circuit by J.~S.~Kilby
%  of Texas Instruments \cite{kilby}.
% His first chip is shown in Fig.~I. For comparison,
% Fig.~I.2 shows a modern microprocessor chip, \cite{beren}.
% 
% 
% This is the introduction.
% This is the introduction.
% This is the introduction.
% This is the introduction.
% This is the introduction.
% This is the introduction.

% \begin{equation}
% ABC {\cal DEF} \alpha\beta\Gamma\Delta\sum^{abc}_{def}
% \end{equation}


% \begin{chapreferences}{3.}
% \bibitem{zkilby}J. S. Kilby,
% ``Invention of the Integrated Circuit,'' {\it IEEE Trans. Electron Devices,}
% {\bf ED-23,} 648 (1976).
% 
% \bibitem{zhamming}R. W. Hamming,
%                  {\it Numerical Methods for Scientists and 
%                  Engineers}, Chapter N-1, McGraw-Hill, 
%                  New York, 1962.
% 
% \bibitem{zHu}J. Lee, K. Mayaram, and C. Hu, ``A Theoretical
%                Study of Gate/Drain Offset in LDD MOSFETs''
%                      {\it IEEE Electron Device Lett.,} {\bf EDL-7}(3). 152 
%                      (1986).
% \end{chapreferences}
% \end{introduction}


% \part[Submicron Semiconductor Manufacture]
% {Submicron Semiconductor\\ Manufacture}

